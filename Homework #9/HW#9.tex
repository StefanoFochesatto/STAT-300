%%%%%%%%%%%%%%%%%%%%%%%%%%%%%%%%%%%%%%%%%%%%%%%%%%%%%%%%%%%%%%%%%%%%%%%%%%%%%%%%%%%%%%%
%%%%%%%%%%%%%%%%%%%%%%%%%%%%%%%%%%%%%%%%%%%%%%%%%%%%%%%%%%%%%%%%%%%%%%%%%%%%%%%%%%%%%%%
% 
% This top part of the document is called the 'preamble'.  Modify it with caution!
%
% The real document starts below where it says 'The main document starts here'.

\documentclass[12pt]{article}
\usepackage{graphicx}
\usepackage{float}
\usepackage{amssymb,amsmath,amsthm}
\usepackage[top=1in, bottom=1in, left=1.25in, right=1.25in]{geometry}
\usepackage{fancyhdr}
\usepackage{enumerate}
\usepackage{listings}
% Comment the following line to use TeX's default font of Computer Modern.
\usepackage{times,txfonts}

\newtheoremstyle{homework}% name of the style to be used
  {18pt}% measure of space to leave above the theorem. E.g.: 3pt
  {12pt}% measure of space to leave below the theorem. E.g.: 3pt
  {}% name of font to use in the body of the theorem
  {}% measure of space to indent
  {\bfseries}% name of head font
  {:}% punctuation between head and body
  {2ex}% space after theorem head; " " = normal interword space
  {}% Manually specify head
\theoremstyle{homework} 

% Set up an Exercise environment and a Solution label.
\newtheorem*{exercisecore}{Exercise \@currentlabel}
\newenvironment{exercise}[1]
{\def\@currentlabel{#1}\exercisecore}
{\endexercisecore}

\newcommand{\localhead}[1]{\par\smallskip\noindent\textbf{#1}\nobreak\\}%
\newcommand\solution{\localhead{Solution:}}

%%%%%%%%%%%%%%%%%%%%%%%%%%%%%%%%%%%%%%%%%%%%%%%%%%%%%%%%%%%%%%%%%%%%%%%%
%
% Stuff for getting the name/document date/title across the header
\makeatletter
\RequirePackage{fancyhdr}
\pagestyle{fancy}
\fancyfoot[C]{\ifnum \value{page} > 1\relax\thepage\fi}
\fancyhead[L]{\ifx\@doclabel\@empty\else\@doclabel\fi}
\fancyhead[C]{\ifx\@docdate\@empty\else\@docdate\fi}
\fancyhead[R]{\ifx\@docauthor\@empty\else\@docauthor\fi}
\headheight 15pt

\def\doclabel#1{\gdef\@doclabel{#1}}
\doclabel{Use {\tt\textbackslash doclabel\{MY LABEL\}}.}
\def\docdate#1{\gdef\@docdate{#1}}
\docdate{Use {\tt\textbackslash docdate\{MY DATE\}}.}
\def\docauthor#1{\gdef\@docauthor{#1}}
\docauthor{Use {\tt\textbackslash docauthor\{MY NAME\}}.}
\makeatother

% Shortcuts for blackboard bold number sets (reals, integers, etc.)
\newcommand{\Reals}{\ensuremath{\mathbb R}}
\newcommand{\Nats}{\ensuremath{\mathbb N}}
\newcommand{\Ints}{\ensuremath{\mathbb Z}}
\newcommand{\Rats}{\ensuremath{\mathbb Q}}
\newcommand{\Cplx}{\ensuremath{\mathbb C}}
%% Some equivalents that some people may prefer.
\let\RR\Reals
\let\NN\Nats
\let\II\Ints
\let\CC\Cplx

%%%%%%%%%%%%%%%%%%%%%%%%%%%%%%%%%%%%%%%%%%%%%%%%%%%%%%%%%%%%%%%%%%%%%%%%%%%%%%%%%%%%%%%
%%%%%%%%%%%%%%%%%%%%%%%%%%%%%%%%%%%%%%%%%%%%%%%%%%%%%%%%%%%%%%%%%%%%%%%%%%%%%%%%%%%%%%%
% 
% The main document start here.

% The following commands set up the material that appears in the header.
\doclabel{Stat 300: Homework 9}
\docauthor{Stefano Fochesatto}
\docdate{\today}

\begin{document}

\begin{exercise}{5.74} In  an  area  having  sandy  soil,  50  small  trees  of  a  certain  type were planted,
   and another 50 trees were planted in an area having clay soil. Let $X = $ the number of trees planted in  sandy  
   soil  that  survive  1  year  and  $Y = $  the  number  of  trees planted in clay soil that survive 1 year.
    If the probability that a tree planted in sandy soil will survive 1 year is .7 and the probability of 1-year survival 
  in clay soil is .6, compute an approximation to $P(-5 \le X-Y \le 5)$ (do not bother with the continuity correction).\\

  \solution Recall that by the CLT since the number of trees is $n = 50 > 30$ we know that $X$ and $Y$ are approximately normal.
  Since $X_i \sim binom(50, .7)$ and $Y_i \sim binom(50, .6)$ we get,
  \begin{equation*}
    \mu_{X} = 50(.7) = 35,
  \end{equation*}  
  \begin{equation*}
    \mu_{Y} = 50(.6) = 30.
  \end{equation*}  
Similarly we can calculate variance of each sample variable,
\begin{equation*}
  \rho_{X}^2 = 50(.7)(.3) = 10.5,
\end{equation*}  
\begin{equation*}
  \rho_{Y}^2 = 50(.6)(.4) = 12.
\end{equation*} 
Calculating the $E(X - Y)$ and $V(X - Y)$,
\begin{equation*}
  E(X - Y) = E(X) - E(Y) = 35 - 30 = 5
\end{equation*}
\begin{equation*}
  V(X - Y) = V(X)+ (-1)^2V(Y) = 10.5 + 12 = 22.5.
\end{equation*}
Then finally we can calculate the probability by standardizing $X - Y$,
\begin{align*}
  P(-5 \le X-Y \le 5) &= P(\dfrac{0 - 5}{\sqrt{22.5}} \le \dfrac{(X-Y) - 5}{\sqrt{22.5}} \le 0)\\
  &= P(-2.11 \le Z \le 0),\\
  & = P(Z \le 0) - P(Z \le -2.11),\\
  &=.4826.
\end{align*}

\end{exercise}
\vspace{.5in}





\begin{exercise}{5.92} 
  \begin{enumerate}
    \item Show that $Cov(X, Y + Z) = Cov(X,Y) + Cov(X, Z)$.\\
    
    \solution By the definition of Covariance, we have the following formula,
    \begin{equation*}
      Cov(X,Y) = E(XY) - E(X)E(Y).
    \end{equation*}
    Applying this formula and using the linearity of expectations we get the following,
    \begin{align*}
      Cov(X, Y + Z) &= E[X(Y + Z)] - E(X)E(Y + Z),\\
      &= E[XY + XZ] - E(X)E(Y + Z),\\
      &= E(XY) + E(XZ) - E(X)E(Y + Z),\\
      &= E(XY) + E(XZ) - E(X)E(Y) - E(X)E(Z),\\
      &= E(XY) - E(X)E(Y) + E(XZ) - E(X)E(Z),\\
      &= Cov(XY) + Cov(XZ).\\
    \end{align*}
    \vspace{.25in}


    \item Let $X_1$ and $X_2$ be quantitative and verbal scores on one aptitude exam, and let $Y_1$ and $Y_2$
      be  corresponding  scores  on  another  exam. If  $Cov(X_1,  Y_1) = 5$, $Cov(X_1, Y_2) = 1$, $Cov(X_2, Y_1) = 2$, 
      and $Cov(X_2, Y_2) = 8$, what is the covariance between the two total scores $X_1 + X_2$ and $Y_1 + Y_2$?\\

      \solution By the previous problem we know that,
      \begin{equation*}
        Cov((X_1 + X_2), (Y_1 + Y_2)) = Cov((X_1 + X_2), Y_1) + Cov((X_1 + X_2), Y_2). 
      \end{equation*}
      Applying the formula again we get that,
      \begin{equation*}
        Cov((X_1 + X_2), (Y_1 + Y_2)) = Cov(X_1, Y_1) + Cov(X_2, Y_1) + Cov(X_1, Y_2) + Cov(X_2, Y_2). 
      \end{equation*}
      By substitution we know that,
      \begin{equation*}
        Cov((X_1 + X_2), (Y_1 + Y_2)) = 5 + 2 + 1 + 8 = 16.
      \end{equation*}
    \end{enumerate}
\end{exercise}
\vspace{.5in}








\begin{exercise}{6.2} The National   Health   and   Nutrition   Examination   Survey   (NHANES)   collects 
demographic,   socioeco­nomic,  dietary,  and  health­related  information  on  an  annual  basis.  Here  is  a  
sample  of  20  observations  on  HDL cholesterol level (mg/dl) obtained from the 2009–2010  survey  
(HDL  is  “good”  cholesterol;  the  higher  its  value, the lower the risk for heart disease)\\
\begin{enumerate}
  \item Calculate a point estimate of the population mean $HDL$ cholesterol level.\\ 
  \solution A sample mean, $\overline{x}$ is a point estimate of the population mean.\\
  
  \textbf{Console:}
  \begin{center}
    \lstinputlisting{mean.txt}
  \end{center}
  \vspace{.25in}


  \item Making no assumptions about the shape of the population distribution, calculate a point estimate  
of the value that separates the largest $50\%$ of HDL levels from the smallest $50\%$\\ 
  \solution By definition we know that the median is the value that separates the largest $50\%$ and smallest
  $50\%$. The point estimate for the population median is the sample median.\\
  
  \textbf{Console:}
  \begin{center}
    \lstinputlisting{median.txt}
  \end{center}



  \item  Calculate a point estimate of the population standard deviation.\\
  
  \solution The sample variance $S^2$ is calculated by,
  \begin{equation*}
    S^2 = \dfrac{\sum_{i = 1}^{20}(x_i - \overline{X})^2}{20 - 1}.
  \end{equation*}
Taking the square root we get the sample standard deviation, through r we see,\\
  
\textbf{Console:}
\begin{center}
  \lstinputlisting{stddev.txt}
\end{center}
\vspace{.25in}

\item An HDL level of at least 60 is considered desirable as it corresponds to a significantly lower risk 
of heart disease.  Making  no  assumptions  about  the  shape  of  the population distribution, estimate the
 proportion $p$ of the population having an HDL level of at least 60.\\
 \solution Using our sample data we can estimate that approximately $20\%$ of the values are above 60.\\ 

 \textbf{Console:}
 \begin{center}
   \lstinputlisting{60.txt}
 \end{center}
\end{enumerate}
\end{exercise}
\vspace{.5in}













\begin{exercise}{6.4} Prior  to  obtaining  data,  denote  the  beam  strengths  by  $X_1, ... ,X_m$   
  and  the  cylinder  strengths  by  $Y_1,  . . .  , Y_n.$ Suppose that the $X_i’s$ constitute a random sample from
  a  distribution  with  mean  $\mu_1$  and  standard  deviation  $\sigma_1$ and  that  the  $Y_i’s$  form  a random  sample  (independent  of the $X_i’s$)
  from another  distribution with mean $\mu_2$ and standard deviation $\sigma_2$.\\

  \begin{enumerate}
    \item Use  rules  of  expected  value  to  show  that $\overline{X} - \overline{Y}$ is an unbiased estimator of $\mu_1 - \mu_2$.
    Calculate the estimate for the given data.\\
    \solution From the rules of expected values we know,
    \begin{align*}
      E(\overline{X} - \overline{Y}) &= E(\overline{X}) - E(\overline{Y}),\\
       &= \dfrac{1}{m} \sum _{i = 1}^m E(X_i) - \dfrac{1}{n} \sum _{i = 1}^n E(Y_i),\\ 
       &= \dfrac{1}{m}mE(X_1) - \dfrac{1}{n}nE(Y_1),\\ 
       &= \mu_1 - \mu_2. 
    \end{align*}
    Therefore $\overline{X} - \overline{Y}$ is an unbiased estimator of $\mu_1 - \mu_2$. Estimating $\mu_1 - \mu_2$ with r,\\
    \textbf{Console:}
    \begin{center}
      \lstinputlisting{meanminusmean.txt}
    \end{center}
  \vspace{.25in}

  \item  Use  rules  of  variance  from  Chapter  5  to  obtain  an  expression  for  the  variance  and 
   standard  deviation  (standard error) of the estimator in part (a), and then compute the estimated standard error.\\
   \solution Since the $X$ and $Y$ are independent we get the following through the rule of variances,
   \begin{align*}
     V(\overline{X} - \overline{Y}) &= V(\overline{X}) +  V(\overline{Y}),\\
     &=\dfrac{1}{m^2} \sum _{i = 1}^m V(X_i) + \dfrac{1}{n^2} \sum _{i = 1}^n V(Y_i),\\
     &=\dfrac{1}{m^2}mV(X_1) + \dfrac{1}{n^2}nV(Y_1),\\
     &=\dfrac{\sigma_1^2}{m} + \dfrac{\sigma_2^2}{n}.\\
   \end{align*}
   Using $r$ to estimate $\dfrac{\sigma_1^2}{m} + \dfrac{\sigma_2^2}{n}$ we get, the sample variience and the square
   root gives us the standard deviation,\\
   \textbf{Console:}
   \begin{center}
     \lstinputlisting{varstd.txt}
   \end{center}
   \vspace{.25in}


   \item Calculate a  point  estimate  of  the  ratio  $\frac{\sigma_1}{\sigma_2}$ of the two standard deviations.\\
   \solution We use the sample variance $S^2$ as a point estimate of the variance so we can get the standard deviation by 
   simply square rooting the sample variance then computing the ratio. \\

  \textbf{Console:}
  \begin{center}
   \lstinputlisting{stdratio.txt}
  \end{center}
\end{enumerate}
\end{exercise}
\vspace{.5in}




\begin{exercise}{6.10}  Using a long rod that has length $\mu$, you are going to lay out  a  square  plot  in  which 
   the  length  of  each  side  is  $\mu$. Thus the area of the plot will be $\mu^2$. However, you do not know the value
    of $\mu$, so you decide to make $n$ indepen­dent measurements $X_1, X_2,  ... , X_n$ of the length. 
    Assume that  each  $X_i$  has  mean  $\mu$  (unbiased  measurements)  and  variance $s^2$\\
    \begin{enumerate}
      \item Show that $\overline{X}^2$ is not an unbiased estimator for $\mu^2$.\\
      \solution By our variance and expected value definition of $E(\overline{X}^2)$, 
      \begin{equation*}
        E(\overline{X}^2) = V(\overline{X}) - E(\overline{X})^2 = \dfrac{\sigma^2}{n} + \mu^2.
      \end{equation*}
      Therefore since,
      \begin{equation*}
        E(\overline{X}^2) \neq \mu^2
      \end{equation*}
      we know that $E(\overline{X}^2)$ is a biased estimator.
      \vspace{.25in}


      \item   For  what  value  of  $k$  is  the  estimator  $\overline{X}^2 - kS^2$ an unbiased for $m^2$?\\
      \solution By the linearity of the expected value we know that,
      \begin{equation*}
        E(\overline{X}^2 - kS^2) = E(\overline{X}^2) - kE(S^2).  
      \end{equation*}
      By the previous problem and since $S^2$ is an unbiased estimator for $\sigma^2$ we know that,
      \begin{equation*}
        E(\overline{X}^2 - kS^2) = \dfrac{\sigma^2}{n} + \mu^2 - k\sigma^2.
      \end{equation*}
      Setting the right hand side to $\mu^2$ and solving for $k$,
      \begin{align*}
        \dfrac{\sigma^2}{n} + \mu^2 - k\sigma^2 &= \mu^2,\\
        k &= \dfrac{1}{n}.
      \end{align*}
    \end{enumerate}
\end{exercise}
\vspace{.5in}



\begin{exercise}{6.12} Suppose a certain type of fertilizer has an expected yield per  acre  of  $\mu_1$  with
    variance  $\sigma^2$,  whereas  the  expected  yield  for  a  second  type  of  fertilizer  is  $\mu_2$  with  the 
    same  variance $\sigma^2$. Let $S^2_1$ and $S^2_2$ denote the sample variances of yields based on sample sizes $n_1$ and $n_2$, 
    respectively, of the two fertilizers. Show that the pooled (combined) estimator is unbiased,
    \begin{equation*}
      \hat{\sigma}^2 = \dfrac{(n_1 - 1)S^2_1 + (n_2 - 1)S^2_2}{(n_1 + n_2 - 2)}.
    \end{equation*}
    \solution Through the linearlity of expectations we can pull all the constants out and separate the sample variences. 
    We can also substitute $\sigma^2$ since $S^2_i$ are unbiased estimators,
    \begin{align*}
      E(\dfrac{(n_1 - 1)S^2_1 + (n_2 - 1)S^2_2}{(n_1 + n_2 - 2)}) &= \dfrac{(n_1 - 1)}{(n_1 + n_2 - 2)}E(S^2_1) + \dfrac{(n_2 - 1)}{(n_1 + n_2 - 2)}E(S^2_2),\\
      &= \dfrac{(n_1 + n_2 - 2)}{(n_1 + n_2 - 2)}\sigma^2,\\
      &= \sigma^2.
    \end{align*}
    Thus our estimator is unbiased. 
\end{exercise}

\end{document}